\section{基于深度神经网络的游戏推荐算法}

\subsection{word2vec}

word2vec是一种比较流行的词嵌入的方法,
通过将单词转换为定长的向量以方便进行计算。
相对于One-Hot编码来说,
没有了过于稀疏的问题,
同时也可以更好地表达词与词之间的相似或类比关系。

word2vec包含有两个模型,
即跳元模型\cite{mikolovDistributedRepresentationsWords2013}
(Skip-gram)
与连续词袋模型\cite{mikolovEfficientEstimationWord2013}
(Continuous Bag of Words),
模型结构如\cref{fig:word2vec}所示。

\begin{figure}[!htbp]
    \centering
    \includegraphics[width=0.8\textwidth]{images/word2vec.pdf}
    \caption{word2vec模型}
    \label{fig:word2vec}
\end{figure}

\subsubsection{跳元模型}

假设一个单词序列$W = \left\{w_1, w_2, w_3, \ldots, w_T\right\}$,
中心词为$w_t$,
上下文窗口为$2$。
则给定中心词$w_t$,
生成上下文词$w_{t-2}, w_{t-1}, w_{t+1}, w_{t+2}$的条件概率为:
\begin{equation}
    \label{eq:con}
    P(w_{t-2}, w_{t-1}, w_{t+1}, w_{t+2}|w_t)
\end{equation}

若上下文词是在给定中心词的情形下满足条件独立性,
则\cref{eq:con}可以改写为:
\begin{equation}
    P(w_{t-2}|w_t)P(w_{t-1}|w_t)P(w_{t+1}|w_t)P(w_{t+2}|w_t)
\end{equation}

令$\mathbf{v}_{w_t}\in\mathbb{R}^d$和
$\mathbf{u}_{w_t}\in\mathbb{R}^d$
表示$w_t$分别作为中心词和上下文词时的两个$d$维向量。
给定中心词$w_I$,
生成的上下文词$w_O$
的条件概率可以通过softmax函数定义:
\begin{equation}
    P\left(w_O|w_I\right) = \frac{\exp\left(\mathbf{u}_{w_O}^T\mathbf{v}_{w_I}\right)}{\sum_{w_t\in W}\exp\left(\mathbf{u}_{w_t}^T\mathbf{v}_{w_I}\right)}
\end{equation}

由于跳元模型是依据条件概率所构建的,
因此我们可以通过极大似然估计来作为模型优化方向,
即:
\begin{equation}
    \frac{1}{T}\sum^T_{t=1}\sum_{-c\leqslant j\leqslant c, j\neq 0}\log\left(w_{t+j|w_t}\right)
\end{equation}

其中$c$为上下文窗口的大小。

\subsubsection{连续词袋模型}

连续词袋模型存在多个上下文词,
故需要在计算条件概率时对上下文词向量进行平均。
即令$\mathbf{v}_{w_t}\in\mathbb{R}^d$和
$\mathbf{u}_{w_t}\in\mathbb{R}^d$
表示$w_t$分别作为中心词和上下文词时的两个$d$维向量。
给定上下文词$w_{O_1}, w_{O_2}, \ldots, w_{O_{2c}}$
生成任意中心词$w_I$的条件概率可以表示为:
\begin{equation}
    P\left(w_I|w_{O_1}, \ldots, w_{O_{2c}}\right) = \frac{\exp\left(\frac{1}{2c}\mathbf{v}_I^T\left(\mathbf{u}_{O_1}+\ldots+\mathbf{u}_{O_{2c}}\right)\right)}{\sum_{w_t\in W}\exp\left(\frac{1}{2c}\mathbf{v}_{w_t}\left(\mathbf{u}_{O_1}+\ldots+\mathbf{u}_{O_{2c}}\right)\right)}
\end{equation}

由此,
连续词袋模型的损失函数可以定义为:
\begin{equation}
    -\sum_{t=1}^T\log P\left(w_t|w_{t-c}, \ldots, w_{t-1}, w_{t+1}, \ldots, w_{t+c}\right)
\end{equation}