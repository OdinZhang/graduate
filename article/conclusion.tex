\section{总结与展望}

\subsection{本文工作总结}

本文主要基于深度学习相关技术对Steam游戏进行相关的
推荐算法进行研究,
利用神经网络对于特征提取的自动化优势,
融合FM、
Word2Vec等相关技术,
在保证了计算效率的同时兼顾了预测的准确率.
本文的主要工作如下:

首先,
介绍了游戏推荐技术的研究背景以及国内外的研究现状,
对本文的工作提出了一定的可行性分析.
其次,
对相关游戏推荐算法进行了简要的介绍,
对本文即将使用的技术做了铺垫.
之后则对Word2Vec以及FM技术和神经网络技术进行介绍,
扫清了本文的相关技术障碍,
最后对模型进行设计以及训练,
并进行相关结果的分析.

\subsection{未来工作展望}

虽然本文的研究告了一段落,
但是仍然有许多的不足,
例如用户ID,
用户名等没有进行特征提取,
可能会影响到最终模型的性能.
同时对于用户已拥有游戏没有纳入考虑,
由于用户数据集的限制,
无法对用户已拥有游戏的游玩时长进行计算,
造成了大量数据的浪费.
对于用户评论数据,
由于评论语种的限制,
无法对其进行自然语言处理的工作,
如果有一种足够好的语种分类的算法,
可以对其利用TextCNN等一系列NLP处理,
以获得更好的推荐效果.
同时,
可以对用户的评论时间以及评论编辑时间进行相应的处理,
以期获得用户偏好的相应变化.
若能在上述方面取得相应的改进并融合其他推荐算法,
在实际应用中应当能取得较好的结果.