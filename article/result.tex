\section{系统实验与结果分析}

\subsection{数据处理}

\subsubsection{数据集的来源}

对于游戏推荐算法的研究较少,
数据集的获取也较为困难,
为了兼顾用户数据与游戏数据的广泛性与准确性,
最终选择加利福尼亚大学圣迭戈分校(UCSD)
的Steam游戏数据集
(\url{https://cseweb.ucsd.edu/~jmcauley/datasets.html#steam_data})。

该数据集的基本统计情况如\cref{tb:dataset}所示。

\begin{table}[!htbp]
	\begin{center}
		\caption{Steam数据集基本统计情况}\label{tb:dataset}
		\begin{tabular}{cccc}
			\toprule
			条目 & 数量        & 条目 & 数量     \\
			\midrule
			评论 & 7,793,069 & 项目 & 15,474 \\
			用户 & 2,567,538 & 合集 & 615    \\
			\bottomrule
		\end{tabular}
	\end{center}
\end{table}

\subsubsection{数据集的预处理}

由于数据集分别包含了评论数据、
用户数据、
游戏数据、
合集数据等。
为了更方便地使用数据集进行训练,
选择将澳大利亚的用户数据、
澳大利亚用户评论数据、
游戏数据进行内连接。

鉴于数据集中有一定数据的缺失,
选择将例如游戏制造商、
游戏发行商、
游戏标题等数据填充为\verb|NONE|,
将游戏标签、
游戏风格、
游戏类型等修改为空列表,
游戏价格填充为\verb|0|。

对数据集处理后的可用条目为53,973条。
选择将数据进行8--2划分,
分别作为训练集和验证集。