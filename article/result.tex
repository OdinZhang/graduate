\section{系统实验与结果分析}

\subsection{实验环境}

本实验主要使用PyTorch框架实现,
使用了Pandas、
NumPy、
gensim等第三方库,
其详细配置如\cref{tb:device}所示.

\begin{table}[!htbp]
  \begin{center}
    \caption{实验环境配置}\label{tb:device}
    \begin{tabular}{cc}
      \toprule
      实验环境    & 规格                                          \\
      \midrule
      操作系统    & Ubuntu Desktop 20.04 LTS                    \\
      运行内存    & 24GB                                        \\
      CPU     & Intel (R) Core (TM) i7--8750H CPU @ 2.20GHz \\
      GPU     & NVIDIA GeForce GTX 1060 6G                  \\
      PyTorch & 1.11.0                                      \\
      CUDA    & 11.3                                        \\
      Python  & 3.9.5                                       \\
      \bottomrule
    \end{tabular}
  \end{center}
\end{table}

\subsection{数据处理}

\subsubsection{数据集的来源}

对于游戏推荐算法的研究较少,
数据集的获取也较为困难,
为了兼顾用户数据与游戏数据的广泛性与准确性,
最终选择加利福尼亚大学圣迭戈分校(UCSD)
的Steam游戏数据集
(\url{https://cseweb.ucsd.edu/~jmcauley/datasets.html#steam_data}).

该数据集的基本统计情况如\cref{tb:dataset}所示.

\begin{table}[!htbp]
  \begin{center}
    \caption{Steam数据集基本统计情况}\label{tb:dataset}
    \begin{tabular}{cccc}
      \toprule
      条目 & 数量        & 条目 & 数量     \\
      \midrule
      评论 & 7,793,069 & 项目 & 15,474 \\
      用户 & 2,567,538 & 合集 & 615    \\
      \bottomrule
    \end{tabular}
  \end{center}
\end{table}

\subsubsection{数据的展示}

该数据集主要有五个文件,
分别为澳大利亚用户评论,
澳大利亚用户项目,
合集数据、
Steam游戏数据以及Steam用户评论数据,
均使用Json文件.
本文将仅使用澳大利亚用户相关数据以及游戏相关数据进行训练.

澳大利亚用户评论数据格式不完整展示为:
\begin{minted}[linenos,
		numbersep=5pt,
		frame=lines,
		framesep=2mm]{json}
{
  "reviews": [
    {
      "funny": "",
      "helpful": "No ratings yet",
      "review": "It's unique and worth a playthrough.",
      "recommend": true,
      "item_id": "22200",
      "last_edited": "",
      "posted": "Posted July 15, 2011."
    },
  ],
  "user_id": "76561197970982479",
  "user_url": "http://steamcommunity.com/profiles/
                                    76561197970982479"
}
\end{minted}

澳大利亚用户项目数据格式不完整展示为:
\begin{minted}[linenos,
    numbersep=5pt,
    frame=lines,
    framesep=2mm]{json}
{
  "steam_id": "76561197970982479",
  "items": [
    {
      "item_id": "10",
      "item_name": "Counter-Strike",
      "playtime_2weeks": 0,
      "playtime_forever": 6
    },
  ],
  "items_count": 277,
  "user_id": "76561197970982479",
  "user_url": "http://steamcommunity.com/profiles/
                                    76561197970982479"
}
\end{minted}

游戏数据格式不完整展示为:
\begin{minted}[linenos,
    numbersep=5pt,
    frame=lines,
    framesep=2mm]{json}
{
  "publisher": "Kotoshiro",
  "genres": ["Action", "Casual", "Indie", "Simulation", 
                                                "Strategy"],
  "app_name": "Lost Summoner Kitty",
  "title": "Lost Summoner Kitty",
  "url": "http://store.steampowered.com/app/761140/
                                        Lost_Summoner_Kitty/",
  "release_date": "2018-01-04",
  "tags": ["Strategy", "Action", "Indie", "Casual", 
                                                "Simulation"],
  "discount_price": 4.49,
  "reviews_url": "http://steamcommunity.com/app/761140/
                        reviews/?browsefilter=mostrecent&p=1",
  "id": "761140",
  "price": 4.99,
  "early_access": false,
  "specs": ["Single-player"],
  "developer": "Kotoshiro"
}
\end{minted}

\subsubsection{数据集的预处理}

由于数据集分别包含了评论数据、
用户数据、
游戏数据、
合集数据等.
为了更方便地使用数据集进行训练,
选择将澳大利亚的用户数据、
澳大利亚用户评论数据、
游戏数据进行内连接.

处理之后用于训练的数据不完整展示为:
\begin{minted}[linenos,
    numbersep=5pt,
    frame=lines,
    framesep=2mm]{json}
{
  "game_publisher": "Tripwire Interactive",
  "game_genres": ["Action"],
  "game_name": "Killing Floor",
  "game_title": "Killing Floor",
  "game_release_date": "2009-05-14",
  "game_tags": [
    "FPS", "Zombies", "Co-op",
  ],
  "game_id": "1250",
  "game_price": 19.99,
  "game_earlt_access": false,
  "game_specs": [
    "Single-player", "Multi-player", "Co-op",
  ],
  "game_dev": "Tripwire Interactive",
  "user_id": "76561197970982479",
  "user_steam_id": "76561197970982479",
  "user_items_count": 277,
  "rec": true,
  "review_posted": "Posted November 5, 2011."
}
\end{minted}

鉴于数据集中有一定数据的缺失,
选择将例如游戏制造商、
游戏发行商、
游戏标题等数据填充为\verb|NONE|,
将游戏标签、
游戏风格、
游戏类型等修改为空列表,
游戏价格填充为\verb|0|.

之后对密集数据进行标准化,
标准化的公式为:
\begin{equation}
  \hat{x} = \frac{x-\mu}{\sigma}
\end{equation}

其中$\mu$为数据的平均值,
$\sigma$为数据的方差.

对数据集处理后的可用条目为53,973条.
选择将数据进行8--2划分,
分别作为训练集和验证集.

\subsubsection{Word2Vec到Tag2Vec的迁移}

由于Word2Vec本质上属于是利用上下文数据来进行词嵌入,
对于tag来说可以认为每一条目的标签均为一个句子,
可以根据tag的共现规律进行Word2Vec的训练,
可以很好地解决OneHot编码过于稀疏的问题.
Word2Vec相较于FM等算法其计算性能较高,
可以很好地应用于标签的特征提取.
同时,
对于训练数据而言,
每一条目的标签不多,
不需要设置过大的窗口,
且几乎没有数据的损失.
因此可以作为模型的预训练部分.

\subsection{模型设计\label{sec:design}}

\subsubsection{FM模块设计}

FM模块主要将离散数据分别交给嵌入层和全连接层进行处理,
其中全连接层输出神经元为$1$,
之后将嵌入层的数据交由FM层处理后与线性层结果相加,
最后与嵌入层的数据拼接后输出.
模块的设计主要见\cref{fig:fm}.

\begin{figure}[!htbp]
  \centering
  \includegraphics[width=.9\textwidth]{images/fm.pdf}
  \caption{FM模块设计}\label{fig:fm}
\end{figure}

\subsubsection{Word2Vec模块设计}

Word2Vec模块主要在预训练阶段把标签数据进行训练,
对于数据集中的风格、
标签、
类型等信息分别进行训练,
并得出维度为$50$的向量,
对每一相应的标签数据对应的向量相加,
将其作为相应的向量,
拼接后作为输出.

\subsubsection{深度神经网络模块设计}

深度神经网络模块的输入数据为$454$维,
首先将数据通过输出为$454$维的全连接层进行训练,
之后通过维度为$150$的全连接层训练,
第三层则为输出为$50$维的全连接层,
第四层和第五层全连接层的输出维度为$10$和$1$,
为了优化计算性能以及获得较好的训练效果,
全连接层之间的激活函数均选择使用为ReLU,
第五层和输出间的激活函数为Sigmoid.
网络的设计可见\cref{fig:deep}.

\begin{figure}[!htbp]
  \centering
  \includegraphics[width=.9\textwidth]{images/deep.pdf}
  \caption{深度神经网络模块设计}\label{fig:deep}
\end{figure}

\subsection{评价指标}

为了评价建立模型的优劣,
需要建立一个相应的评价体系.
对于本文而言,
其推荐质量的评价主要是通过
对给定数据产生推荐结果的准确性实现.

该推荐算法对游戏的预测主要是对游戏评分的预测,
算法根据用户对游戏评分的多少进行相应的推荐.
因此,
需要一种算法来衡量算法预测评分和用户真实评分的
一种评价指标.
通常的评价指标有:
均方误差损失、
平均绝对值误差损失、
交叉熵损失、
负对数似然函数损失等等.
鉴于该问题适用于二分类问题
(用户评分只有推荐与否),
因此为避免类似于均方误差损失等带来的梯度消失等一系列问题,
因此采用二分类交叉熵损失(BCE).
其计算公式为:
\begin{equation}
  L=-\sum_{i=1}^{N}\left[y_{i} \ln \left(\sigma\left(x_{i}\right)\right)+\left(1-y_{i}\right) \ln \left(1-\sigma\left(x_{i}\right)\right)\right]
\end{equation}

此时有:
\begin{equation}
  \begin{aligned}
    \frac{d L}{d x_{i}} & =\frac{d L}{d \sigma\left(x_{i}\right)} \cdot \frac{d \sigma\left(x_{i}\right)}{d x_{i}}                                                                                         \\
                        & =-\left[\frac{y_{i}}{\sigma\left(x_{i}\right)}+\left(1-y_{i}\right) \frac{-1}{1-\sigma\left(x_{i}\right)}\right] \sigma\left(x_{i}\right)\left(1-\sigma\left(x_{i}\right)\right) \\
                        & =\left(\frac{1-y_{i}}{1-\sigma\left(x_{i}\right)}-\frac{y_{i}}{\sigma\left(x_{i}\right)}\right) \sigma\left(x_{i}\right)\left(1-\sigma\left(x_{i}\right)\right)                  \\
                        & =\left(1-y_{i}\right) \sigma\left(x_{i}\right)-y_{i}\left(1-\sigma\left(x_{i}\right)\right)                                                                                      \\
                        & =\sigma\left(x_{i}\right)-y_{i}
  \end{aligned}
\end{equation}

其中$ \sigma(x)=1/\left(1+e^{-x}\right) $.

此时可以看到采用Sigmoid作为激活函数与
BCE作为损失函数所得到的梯度值正比于预测值与真实值之差.

\subsection{结果分析}

如前文所述,
对数据集进行训练可以得到训练的结果如\cref{fig:loss}所示.

\begin{figure}[!htbp]
  \centering
  \includegraphics[width=.7\textwidth]{images/loss.pdf}
  \caption{模型损失率曲线}\label{fig:loss}
\end{figure}

由图所示,
伴随迭代周期的增加,
模型的平均损失率在不断减小,
在大约$20$次时趋于收敛.
在第$100$次左右时损失率已经接近$0.17$.
可以看出基本符合预期要求,
同时模型的准确率也能维持在90\%以上的水平,
满足对于推荐所需要的准确率.

模型的性能比较如\cref{tb:comp}所示.

\begin{table}[!htbp]
  \begin{center}
    \caption{Steam数据集基本统计情况}\label{tb:comp}
    \begin{tabular}{cccc}
      \toprule
             & Loss   & 准确率 & 训练时间     \\
      \midrule
      DeepFM & $0.16$ & $90.5\%$  & $33$ min \\
      本文模型   & $0.17$ & $93.7\%$  & $46$ min    \\
      \bottomrule
    \end{tabular}
  \end{center}
\end{table}

对比DeepFM模型而言,
本文所提出的模型损失率略高,
同时准确率也有所降低,
但是就性能而言有一定的提升.

\subsection{TopN推荐}

本文使用上述模型对用户与游戏组合进行分别计算后,
按照预测数值大小进行排序,
选取最大的$9$个游戏以及
从其余游戏中加权随机生成一个游戏作为最终的推荐结果,
在保证推荐准确率的同时可以对用户进行一定的新偏好培养.

在对本人以及周围同学进行推荐测试后,
其推荐准确率约为$13.7$\%,
对于其日常使用Steam平台所得的反馈而言,
准确率具有一定的提升,
其推荐性能也在可接受的范围内,
在受测人群有限的情况下,
可以认为本模型初步满足设计需求.
