\section{游戏推荐算法简介}

自上世纪90年代中叶协同过滤算法的提出
\cite{adomaviciusNextGenerationRecommender2005},
使得推荐系统作为一门独立的学科被世人广泛且深入的研究。
在此之后工业界和学术界都提出了一些新的推荐系统的算法。
经过长期的发展,
已经可以给出推荐算法的形式化定义:
令$C$表示为用户集合,
$S$作为所有可作为推荐项目的集合。
在实际的应用生产环境中,
集合$C$和集合$S$都具有相当大的规模。
定义效用函数$u$来计算项目$s$对用户$c$的有用程度,
即$u:C\times S\rightarrow R$。
其中$R$是一个全序集合(即一定范围内的非负整数或实数)。
因此,
对每个用户$c\in C$,
我们期望选出项目$s'\in S$使得可以最大化用户的效用。
形式化表达如下:
\begin{equation}
    \forall c\in C,\; s_c'=\arg \max_{s\in S} u(c,s)
\end{equation}

传统的推荐算法通常被分类为以下三种类型
\cite{balabanovicFabContentbasedCollaborative1997}:
\begin{itemize}
    \item 基于内容的推荐
    \item 基于协同过滤的推荐
    \item 混合推荐
\end{itemize}

\subsection{基于内容的推荐}

基于内容的推荐是较早使用的推荐算法,
其核心思想在于给用户推荐与过去相比较为相似的项目。
即通过对用户已经拥有过的或者评分过的项目的元数据特征进行提取,
计算出用户的偏好,
之后计算用户与项目的相似度并根据相似度的大小进行排序,
排序的结果作为最终的推荐结果。

基于内容的推荐主要依赖于用户的偏好与项目的特征,
不需要其他用户的相关信息,
有效的避免了数据的稀疏性问题,
同时也解决了新项目的冷启动问题。
但是,
基于内容的推荐对特征提取的要求较高,
也很难衡量推荐项目的优劣性,
对于新用户或历史记录较少的用户而言也不够友好,
很难对其进行有效的推荐。

\subsection{基于协同过滤的推荐}

协同过滤算法是Goldberg等人
\cite{goldbergUsingCollaborativeFiltering1992}
在1992年提出的一种推荐算法。
其目前为推荐算法当中较为主流的研究方向,
同时也是应用较为广泛的推荐算法。

协同过滤算法主要可以分为基于用户的协同过滤与基于项目的协同过滤。
基于用户的协同过滤主要为
给用户推荐与其偏好较为相似的用户所喜爱的项目,
相似地,
基于项目的协同过滤则主要为
给用户推荐与其喜欢的项目相似的项目。
即依据用户的历史行为偏好计算用户间的相似度(基于用户的协同过滤)
或项目间的相似度(基于项目的协同过滤),
利用用户的历史行为偏好预测用户未来可能表示偏好的项目进行推荐。
相似度的计算主要可以分为以下三种
\cite{heJiYuJuanJiShenJingWangLuoDeYinLeTuiJianXiTong2019}:
\begin{itemize}
    \item 余弦相似度
    \item 皮尔逊相关系数
    \item 欧几里得距离
\end{itemize}

\subsection{混合推荐}
