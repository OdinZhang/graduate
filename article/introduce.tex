\section{游戏推荐算法简介}

自上世纪90年代中叶协同过滤算法的提出\cite{adomaviciusNextGenerationRecommender2005},
使得推荐系统作为一门独立的学科被世人广泛且深入的研究.
在此之后工业界和学术界都提出了一些新的推荐系统的算法.
经过长期的发展,
已经可以给出推荐算法的形式化定义:
令$C$表示为用户集合,
$S$作为所有可作为推荐项目的集合.
在实际的应用生产环境中,
集合$C$和集合$S$都具有相当大的规模.
定义效用函数$u$来计算项目$s$对用户$c$的有用程度,
即$u:C\times S\rightarrow R$.
其中$R$是一个全序集合(即一定范围内的非负整数或实数).
因此,
对每个用户$c\in C$,
我们期望选出项目$s'\in S$使得可以最大化用户的效用.
形式化表达如下:
\begin{equation}
    \forall c\in C,\; s_c'=\arg \max_{s\in S} u(c,s)
\end{equation}

传统的推荐算法通常被分类为以下几种类型\cite{canoHybridRecommenderSystems2017}:
基于内容的推荐、
基于协同过滤的推荐、
统计推荐、
基于知识的推荐、
混合推荐.

\subsection{基于内容的推荐}

基于内容的推荐是较早使用的推荐算法,
其核心思想在于给用户推荐与过去相比较为相似的项目.
即通过对用户已经拥有过的或者评分过的项目的元数据特征进行提取,
计算出用户的偏好,
之后计算用户与项目的相似度并根据相似度的大小进行排序,
排序的结果作为最终的推荐结果.

基于内容的推荐主要依赖于用户的偏好与项目的特征,
不需要其他用户的相关信息,
有效的避免了数据的稀疏性问题,
同时也解决了新项目的冷启动问题.
但是,
基于内容的推荐对特征提取的要求较高,
也很难衡量推荐项目的优劣性,
对于新用户或历史记录较少的用户而言也不够友好,
很难对其进行有效的推荐.

\subsection{基于协同过滤的推荐}

协同过滤算法是Goldberg等人\cite{goldbergUsingCollaborativeFiltering1992}
在1992年提出的一种推荐算法.
其目前为推荐算法当中较为主流的研究方向,
同时也是应用较为广泛的推荐算法.

协同过滤算法主要可以分为基于用户的协同过滤与基于项目的协同过滤.
基于用户的协同过滤主要为
给用户推荐与其偏好较为相似的用户所喜爱的项目,
相似地,
基于项目的协同过滤则主要为
给用户推荐与其喜欢的项目相似的项目.
即依据用户的历史行为偏好计算用户间的相似度(基于用户的协同过滤)
或项目间的相似度(基于项目的协同过滤),
利用用户的历史行为偏好预测用户未来可能表示偏好的项目进行推荐.

\subsection{统计推荐}

统计推荐使用年龄、
性别、
学历等统计数据,
来为用户打上不同的标签.
这种方案可以很好的解决冷启动问题.
然而,
在当今环境下,
由于在线隐私政策,
很难收集到足够多的统计数据来进行足够好的推荐.
但是统计推荐仍然被用来结合其他推荐算法来获得更好的推荐结果.

\subsection{基于知识的推荐}

基于知识的推荐(KBF)
使用关于用户和项目的知识来得知哪种项目满足用户的需求,
并且生成相应的结果\cite{burkeKnowledgeBasedRecommenderSystems2000}.
KBF可以用于推荐用户购买较少的复杂项目(例如汽车、房子等),
并且表示用户(或价格)的重要限制\cite{felfernigConstraintbasedRecommenderSystems2008}.
在这种拥有较少的用户交互数据的系统中使用协同过滤算法或
基于内容的推荐不够现实.

\subsection{混合推荐}

在现实情况下,
单一的推荐算法往往不能很好的解决实际的问题,
因此,
为了充分利用相关数据,
发挥各种推荐算法的优势,
将两种或多种推荐算法组合起来的推荐技术则被称为混合推荐技术.
常用的混合推荐方式有加权型、
分级型、
瀑布型、
合并型、
特征组合型、
特征递增型、
融入其他因素推荐等等.
下文将对一些常见的组合策略进行介绍\cite{heJiYuJuanJiShenJingWangLuoDeYinLeTuiJianXiTong2019}.

\subsubsection{加权型}

加权型将独立的两个或多个推荐算法的算法通过加权相结合,
并将其作为最终的推荐结果.
通过比较用户的评分与算法预测结果的差异进行权重调整.
对于任意用户$u$,
其对项目$i$的推荐度公式为:
\begin{equation}
    f(u,i)=\sum_{j=1}^n\alpha_j f_j(u,i)
\end{equation}

其中$f_j(u,i)$为不同的推荐算法,
$\alpha_j$为不同推荐算法所对应的权重.
加权型推荐精度较高,
但是权重系数调整相对困难,
较为消耗计算资源,
因此实际应用中使用较少.

\subsubsection{分级型}

分级型通过将不同推荐算法的推荐结果进行层次划分,
其次在相应的场景中选用可信度较高的算法推荐,
最后采用其余算法进行递补推荐,
其公式为:
\begin{equation}
    f(u,i)=\sum_{j=1}^n\beta_j(u,i)f_j(u,i)
\end{equation}

其中$f_j(u,i)$为不同推荐算法,
$\beta_j(u,i)$为相应场景中的对应方法的优劣层级.
分级型相对适合进行TopN类型推荐,
且能兼顾推荐的结果的质量与数量.

\subsubsection{瀑布型}

瀑布型通过将推荐算法视为不同粒度的过滤器,
并将前一个推荐算法的输出作为下一个推荐算法的输入,
通过逐步筛选候选结果以得到高精确度的结果,
其公式为:
\begin{equation}
    f(u,i)=g\left(\sum_{j=1}^n\lambda_j f_j(u,i)\right)
\end{equation}

其中$g(\cdot)$为外层推荐算法,
$\sum_{j=1}^n\lambda_j f_j(u,i)$为前一级推荐结果,
$\lambda_j$代表前一级权重.
通常在设计时,
会将便于计算、
较低区分度的推荐算法放在前一级,
以节省计算资源.
通常瀑布型算法在待推荐项目和所需推荐结果数量相差较大时使用.

\subsubsection{合并型}

合并型采用同时使用
多种推荐方法进行推荐并给出各自的推荐结果以及推荐理由
的方案,
以供用户参考.
该方法的推荐结果较为全面,
种类丰富,
有利于培养用户的新偏好.
