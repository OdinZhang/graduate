\section{系统分析}

\subsection{需求分析}

\subsection{可行性分析}

\subsubsection{社会可行性}

随着游戏行业的发展,
游戏数量越来越多,
用户的评论数据、
游玩数据等也呈爆炸式增长趋势。
同时国内外也拥有大量的推荐算法研究项目,
提供了较为丰富的理论知识以及数据支撑。
道德与法律层面同样符合国家的相关要求。

\subsubsection{技术可行性}

就目前而言,
已经有两款工业界和学术界均较为认可的神经网络框架:
PyTorch和TensorFlow,
同时由于其基于Python语言与C++语言实现,
具有较为广泛的适用性。
对于数据的预处理,
网络的训练以及后续的相关操作均可轻松实现。
同时,
由于其可以使用CUDA或ROCm等显卡加速技术,
对于计算机性能的要求对比纯CPU运算较为节省计算机资源。

算法方面,
本文使用的Word2Vec,
因子分解机FM,
Adam算法等均在学术界以及工业界拥有较为广泛的应用。
同时Python也有大量的相关算法实现模型,
例如Pytorch,Tensorflow,gensim等等。
数据处理方面也有较为成熟的Python第三方库,
例如NumPy,Pandas等。
就本文而言,
上述技术基础较为坚实,
可以轻易编写相关代码,
对所设计的推荐算法进行相关实现。

\subsubsection{操作可行性}

该算法使用Python实现,
核心部分使用PyTorch等相关第三方库,
具有较强的通用性,
同时,
训练好的网络可以使用多种方案进行使用,
在桌面平台、
移动平台等均可使用。