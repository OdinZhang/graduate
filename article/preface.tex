\section{绪论}

\subsection{研究背景与意义}

随着21世纪互联网的高速发展,
数据的产生与获取也越来越方便,
根据相关研究表明\cite{arnevonseeTotalDataVolume2021},
2021年的数据总量被认为是79ZB,
在2022年,全球数据总量将可能达到97ZB。

网络的迅速发展所带来巨大讯息的同时,
对于普通人来说信息过载的问题也将日趋严重。
因此,
如何建立一个有效的数据分类系统成为了一个日益重要的议题。
对于游戏领域,
以目前最大的游戏应用商店Steam来说,
2021年一年所发布的新游戏已经达到了10000部\cite{NumberGamesReleased},
月活用户更是超过1亿,
而且仍有增长的趋势。
因此,
不可避免地需要一种快速、
有效的推荐系统来给每位用户推荐其所偏好的游戏。

\subsection{国内外研究现状}

\subsubsection{国内研究现状}

国内学者对于游戏推荐领域兴趣似乎并不算大,
主要的有俞东进教授及其团队\cite{yuJiYuYinShiFanKuiShuJuDeGeXingHuaYouXiTuiJian2018}以及沙静等人\cite{shaJiYuYinShiFanKuiDeGeXingHuaYouXiTuiJianFangFa2021}基于隐式反馈的游戏推荐系统,
薛梦婷\cite{xieJiYuZhongWenPingLunQingGanQingXiangXingFenXiDeShouYouTuiJianYanJiu2017}基于评论情感倾向性的手游推荐系统,
陈耀旺\cite{chenJiYuShenDuXueXiDeGeXingHuaWangBaYouXiTuiJian2019}等人基于深度学习的游戏推荐系统。
其余学者则多集中在算法的实现方面,
例如使用KNN、
DNN等算法。

相较于国内游戏推荐系统领域的研究匮乏,
其余领域的推荐系统研究则相对较为充分。
例如冯兴杰等人\cite{fengJiYuPingFenJuZhenYuPingLunWenBenDeShenDuTuiJianMoXing2020}基于评分矩阵和评论文本所设计的深度推荐系统,
何蓉\cite{heJiYuJuanJiShenJingWangLuoDeYinLeTuiJianXiTong2019}所提出的基于CNN的音乐推荐系统,
以及吕亚珉\cite{luJiYuFMYuDQNJieHeDeShiPinTuiJianSuanFa2021}等人基于FM和DQN的视频推荐算法等等。

\subsubsection{国外研究现状}

国外学者对游戏推荐系统的研究主要有Gong等人\cite{gongHybridRecommenderSystem2020}提出的混合过滤的Steam游戏推荐系统,
Wang\cite{kangSelfAttentiveSequentialRecommendation2018}等人提出的自我注意的序列化推荐系统以及Pathak\cite{pathakGeneratingPersonalizingBundle2017}等人对于生成Steam游戏合集的推荐系统的研究等等。
而对近似领域的推荐系统的研究则多集中于使用深度学习领域的相关工具,
例如Zhou\cite{zhouCNNRNNBasedIntelligent2021}等人基于CNN、
RNN的媒体推荐系统,
Sahoo\cite{sahooDeepRecoDeepLearning2019}等人基于深度学习的协同过滤推荐系统,
甚至有Wang\cite{wangAdversarialTrainingBasedMean2020}等人基于对抗训练的贝叶斯个性化排序推荐系统等。

\subsection{论文主要研究内容}

由于游戏推荐方面包含用户的游玩时间、
游戏评论与打分以及游戏标签等多种信息,
因此特征分解、
评论情感分析、
深度学习等算法均可用于游戏推荐系统当中。
